\documentclass[12pt]{article}
\usepackage{times}
\usepackage{fullpage}
\usepackage{graphicx}

\setlength{\parindent}{0in}
\setlength{\parskip}{0.1in}

\begin{document}

\thispagestyle{empty}

{\large CS3311 Homework 9} \hfill
Due date: Tuesday, November 6, 2018, 11:59pm\\
\hfill
Submission: Typed, pdf on Canvas (scanned submissions are not allowed)\\
\vspace{-0.1in}
\rule{\textwidth}{0.5mm}

\begin{small}
The answers must be the original work of the author.  While discussion
with others is permitted and encouraged, the final work should be done
individually. You are not allowed to work in groups.  You are allowed to
build on the material supplied in the class. Any other source must be
specified clearly.
\end{small}
\rule{\textwidth}{0.5mm}


{\bf 1.} {\em (20+30 points)} Let $M$ be the following NFA-$\lambda$:

\vspace{-0.15in}

\begin{center}
\includegraphics[width=0.35\textwidth]{nfa-lambda-to-dfa-q.pdf}
\end{center}

\vspace{-0.1in}

{\bf (a)} Give the input transition function (t) for $M$ in tabular form.
Include a column for the $\lambda$-closure of each state. Remember that
each state is a member of its $\lambda$-closure.
\\this is the transition function(not the input transition function)\\
\begin{tabular}{l |c| c| r}
    $\delta$&a&b&$\lambda$\\
    $q_0$&$q_0$ &$\emptyset$ & $q_1,q_3$\\
    $q_1$& $q_2$&$q_4$ &$\emptyset$  \\
    $q_2$&$\emptyset$ &$q_1$ & $\emptyset$ \\
    $q_3$& $q_4$&$q_3$ &$\emptyset$ \\
    $q_4$& $\emptyset$&$\emptyset$ &$\emptyset$  \\

\end{tabular}
\\
\\this is the $\lambda$-transition\\
\begin{tabular}{l|r}
    '$\lambda$-closure'&$\lambda$-closure\\
    $q_0$&$q_0,q_1,q_3$\\
    $q_1$& $q_1$  \\
    $q_2$&$q_2$ \\
    $q_3$& $q_3$ \\
    $q_4$&$q_4$ \\

\end{tabular}
\\
\\this is the input transition\\
\begin{tabular}{l |c|r}
    $t$&a&b\\
    $q_0$&$q_0,q_1,q_2,q_3,q_4$ &$q_3,q_4$ \\
    $q_1$& $q_2$&$q_4$\\
    $q_2$ & $\emptyset$ &$q_1$ \\
    $q_3$& $q_4$&$q_3$  \\
    $q_4$& $\emptyset$&$\emptyset$   \\

\end{tabular}\\
{\bf (b)} Construct a state diagram of a DFA that
is equivalent to $M$. Give the transition function and draw the
state diagram of the equivalent DFA.
\\this is the transition function(not the input transition function)
\begin{tabular}{l |c|c| r}
    '  '&a&b\\
    $q_0$&$q_0$ &$\emptyset$ & $q_0,q_1,q_3$\\
    $q_1$& $q_2$&$q_4$ &$q_1$ \\
    $q_2$&$\emptyset$ &$q_1$ & $q_2$\\
    $q_3$& $q_4$&$q_3$ &$q_3$ \\
    $q_4$& $\emptyset$&$\emptyset$ &$q_4$ \\

\end{tabular}
\\

{\bf 2.} {\em (30+20 points)} Consider the DFA below.

\vspace{-0.1in}

\begin{center}
\includegraphics[width=0.5\textwidth]{dfa-to-minimize.pdf}
\end{center}

{\bf (a)} Construct a two dimensional table where the row and column
headers are the states of the above DFA. Mark each cell with a `1' (or a
higher number representing the iteration number) if the states are
``different.'' Unmarked cells will represent indistinguishable states.

{\bf (b)} Construct a minimized DFA by collapsing (groups of)
indistinguishable states into single states.



\end{document}

